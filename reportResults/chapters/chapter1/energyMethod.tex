\section{Energy Method}
The energy method is applied to find the dynamic equations starting by using excessive coordinates. The potential and kinetic energy for the pendulum is,
\begin{flalign}
  U_p &= mgh                                                           & \unit{\cdot} \\
  T_p &= \tfrac{1}{2} m \dot{x}_p^2 + \tfrac{1}{2} m \dot{y}_p^2 \ \ \ , & \unit{\cdot}
  \label{eq:pendulumEnergy}
\end{flalign}
\begin{where}\\
  \va{ U_p   }{is the potential energy of the pendulum}    {\cdot}
  \va{ T_p   }{is the kinetic energy of the pendulum}      {\cdot}
  \va{ h     }{is the height of the pendulum point mass}   {m}
\end{where}

where the height, $h$, is given in relation to the pendulum at rest at $\theta = \pm \pi n$ by,
\begin{flalign}
  h &= l( 1 + \cos \theta ) \ \ \ . & \unit{\cdot}
  \label{eq:height}
\end{flalign}

The potential and kinetic energy in excessive coordinates for the cart is,
\begin{flalign}
  U_c &= 0           & \unit{\cdot} \\
  T_c &= \tfrac{1}{2} M \dot{x}_c^2   \ \ \ . & \unit{\cdot}
  \label{eq:cartEnergy}
\end{flalign}
\begin{where}
  \va{ U_c               }{is the potential energy of the cart}                    {\cdot}
  \va{ T_c               }{is the kinetic energy of the cart}                      {\cdot}
\end{where}

There is no potential energy for the cart as it is constrained at $y_c=0$.\\
The combined energies for the system in excessive coordinates is then,
\begin{flalign}
  U &= mgl( 1 + \cos \theta ) + 0    & \unit{\cdot} \\
  T &= \tfrac{1}{2} m \dot{x}_p^2 + \tfrac{1}{2} m \dot{y}_p^2  + \tfrac{1}{2} M \dot{x}_c^2 \ \ \ , & \unit{\cdot}
  \label{eq:excessivePotentialAndKinetic}
\end{flalign}
%
Using the coordinate transformation from \autoref{eq:coordinateTransformation} and \ref{eq:transformationDerivatives} to obtain the energies of the system in generalized coordinates,
\begin{flalign}
  \begin{cases}
    U &= mgl( 1 + \cos \theta ) + 0 \\
    T &= \frac{1}{2} m ( \dot{x} - l\cos \theta \dot{\theta} )^2 + \frac{1}{2} m ( -l\sin \theta \dot{\theta} )^2  + \frac{1}{2} M \dot{x}^2 
  \end{cases} && \nonumber
\end{flalign}
\vspace{-14pt}
\begin{flalign}
  \begin{cases}
    U &= mgl( 1 + \cos \theta )     \\
    T &= \frac{1}{2} ( M + m ) \dot{x}^2 - m \dot{x} l \cos \theta \dot{\theta} + \frac{1}{2} m l^2 \dot{\theta}^2 \ \ \ , 
  \end{cases} & \unit{\cdot}
  \label{eq:generalizedPotentialAndKinetic}
\end{flalign}

By \autoref{eq:generalizedPotentialAndKinetic} the Lagrangian becomes,
%
\begin{flalign}
  \cal{L} &= T - U && \nonumber \\ 
  \cal{L} &= \tfrac{1}{2} ( M + m ) \dot{x}^2 - m \dot{x} l \cos \theta \dot{\theta} + \tfrac{1}{2} m l^2 \dot{\theta}^2 - m g l( 1 + \cos \theta ) \ \ \ . & \unit{N \cdot m}
  \label{eq:lagrangian}
\end{flalign}
%
\begin{where}
  \va{ \cal{L}           }{is the Lagrangian}                          {N \cdot m}
\end{where}

From the energy method\fxnote{source and thermonology on energy method} we find the dynamic equations by,
%
\begin{flalign}
  \frac{\partial \cal{L}}{\partial \vec{q}} - \frac{d}{dt}  \frac{\partial \cal{L}}{\partial \vec{\dot{q}}}  &=  0 \ \ \ . &&
  \label{eq:energyMethod}
\end{flalign}
%
\begin{where}
  \va{ \vec{q}         }{is the generalized coordinates, $[\ \theta\ \ x\ ]^{T}$}             {}
  \va{ \vec{\dot{q}}   }{is the generalized velocities, $[\ \dot{\theta}\ \ \dot{x}\ ]^{T}$} {}
\end{where}

In order to include external forces \fxnote{add formal mumbojumbo} d'Alambert principle,
\begin{flalign}
  \frac{d}{dt}  \frac{\partial \cal{L}}{\partial \vec{\dot{q}}} - \frac{\partial \cal{L}}{\partial \vec{q}}  &=  \vec{Q} \ \ \ . &&
  \label{eq:energyMethodWith external forces}
\end{flalign}
%
\begin{where}
  \va{ \vec{Q}   }{is the external forces, $[\ -B_p(\dot{\theta})\ \ F - B_c(\dot{x})\ ]^{T}$}   {}
\end{where}



For yields the final two dynamic equations, one for each of the generalized coordinates,
%
\begin{flalign}
  \begin{cases}
     \frac{d}{dt}  \frac{\partial \cal{L}}{\partial \dot{\theta}} - \frac{\partial \cal{L}}{\partial \theta}  &= m \dot{x} l \sin \theta \dot{\theta} - m l \cos \theta \ddot{x} + m l^2 \ddot{\theta} - m \dot{x} l \sin \theta \dot{\theta} - m g l \sin \theta   \\ %\unit{N \cdot m}  \\
     \frac{d}{dt}  \frac{\partial \cal{L}}{\partial \dot{x}}  - \frac{\partial \cal{L}}{\partial x} &=  ( M + m )\ddot{x} + m l \sin \theta \dot{\theta}^2 - m l \cos \theta \ddot{\theta} - 0 \\ %\unit{N \cdot m}
  \end{cases} \nonumber &&
\end{flalign}
\vspace{-14pt}
\begin{flalign}
  \begin{cases}
    m l^2 \ddot{\theta} - m l \cos \theta \ddot{x} - m g l \sin \theta  = -B_p(\dot{\theta}) & \\                      %\unit{N \cdot m}  \\
    ( M + m )\ddot{x} + m l \sin \theta \dot{\theta}^2 - m l \cos \theta \ddot{\theta}  =  F - B_c(\dot{x})  \ \ \ , &   %\unit{N \cdot m}
  \end{cases} & \unit{\cdot}
  \label{eq:energyDerivedDynamicEquations}
\end{flalign}
%
which is the same result as achieved by Newton's method, as seen by comparing \autoref{eq:generalizedCoordinates} and \ref{eq:energyDerivedDynamicEquations}.




